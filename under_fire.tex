\documentclass{book}

\usepackage[dvipsnames, table]{xcolor}
\usepackage{swrpg}
\usepackage{genesys}
\usepackage{hyperref}
%\geometry{paperheight=13in}
\usepackage{multicol}
\usepackage[noorphans,font=itshape,indentfirst=false,rightmargin=0pt]{quoting}

\swrpgbackground

% starwars dice
\newcommand{\df}{\difficulty}
\newcommand{\stb}{\setback}
\newcommand{\ch}{\challenge}

\renewcommand\Characteristics[6]{
\noindent\hfill
\charPart{BRAWN}{#1}\kern-.1em
\charPart{AGILITY}{#2}\kern-.1em
\charPart{INTELLECT}{#3}\kern-.1em
\charPart{CUNNING}{#4}\kern-.1em
\charPart{WILLPOWER}{#5}\kern-.1em
\charPart{PRESENCE}{#6}%
\hfill\null}

\newcommand{\npc}[9]{
    \begin{GenesysTable}{l} {\Large #1 } \end{GenesysTable} \\ %name
    \Characteristics{#2}{#3}{#4}{#5}{#6}{#7}\\
    \vskip -1em
    \Derived{Wounds}{#8}
    \Derived{Soak}{#9}
}
\newcommand{\npccontinued}[3]{
    \DerivedSplit{Defense}{#1}{#2}{Melee}{Ranged}\\ % defense
    \vskip -2.5em
    #3 %description, abilities, etc
}

\newcommand{\nemesis}[9]{
    \begin{GenesysTable}{l} {\Large #1 [Nemesis] } \end{GenesysTable} \\ %name
    \Characteristics{#2}{#3}{#4}{#5}{#6}{#7}\\
    \vskip -1em
    \Derived{Wounds}{#8}\hskip-.1em
    \Derived{Strain}{#9}\hskip-.1em
}
\newcommand{\nemesiscontinued}[4]{
    \Derived{Soak}{#1}\hskip-.1em
    \DerivedSplit{Defense}{#2}{#3}{Melee}{Ranged}\\ % defense
    \vskip -2.5em
    #4 %description, abilities, etc
}

\title{
Star Wars: Under Fire \\
A DiceyStories.com SWRPG Adventure\\
}
\author{Author: Jennifer Van Boxel (jendefer)\\
        Layout: Dan Van Boxel\\
        Based On: ``First Flight'' by Shannon Phillips \\\url{https://archiveofourown.org/works/7030051}}
\date{\today}


\begin{document}
\maketitle

% Credits:
% Author Jennifer Van Boxel (jendefer)
% Layout DVB
% Based on "First Flight" by Shannon Phillips 
% \url{https://archiveofourown.org/works/7030051})
% Playtesters Desert Rose Solutions
% Information on Sullust drawn from Strongholds of Resistance

\begin{multicols}{2}
\section{Opening Crawl}


It is a time of civil war. The evil GALACTIC EMPIRE continues to strike against the rag-tag REBEL ALLIANCE. During this time of downed ships and lost causes, rebel technicians have created a new type of emergency beacon. Discreet and more powerful than the standard model, these engineering marvels could save countless Rebel lives, if only they get into the right hands before they are needed.

As they follow jobs across the galaxy, a ragtag crew has been delivering boxes of transponders to Rebel operatives and sympathizers. Next on their list is AADLAN MYDER of the Cobalt Front. But finding this contact on the lava-licked world of SULLUST may prove a challenge….



\section{Adventure Overview }

The Player Characters are on Sullust to deliver a small crate of Rebellion emergency beacons. The Cobalt Front is not formally a member of the Rebellion; the materials are a gesture of good will to woo them into closer affiliation. If this particular item does not fit with your groups interests, feel free to replace it with an appropriate substitute (weapons, medicines, other tech, etc.).

The Player Characters could be Rebel operatives in an Age of Rebellion game, smugglers from an Edge of the Empire Game, or symapthetic Force fugitives from a Force and Destiny game, whatever works for your group. Similarly, this adventure can be set before the Battle of Yavin (as is the short story on which it was based) or during the Rebel activities of the original Star Wars trilogy.\\
\emph{Goal}: Deliver transponders to Aadlan Myder of the Cobalt Laborers’ Reformation Front in Pinyumb.\\
\emph{Challenge}: Aadlan Myder must be rescued from a mining accident.

\subsection{Encounter List}
\begin{enumerate}
    \item Arrival on Sullust (information)
    \item Not Home (social)
    \item Finding Parking (exploration/vehicle)
    \item Underground Travel (exploration/combat)
    \item The Miners (problem solving)
    \item Aftermath
\end{enumerate}

\section{Arrival on Sullust}

Note: Sullust is detailed in Strongholds of Rebellion starting on page 36.

\subsection{Airspace}

Description:\\
The ship comes out of hyperspace and approaches Sullust. Half the planet’s surface is covered with belches of ash and toxic gases from its ever-active volcanos; the enormous factories below can be glimpsed through swirls of smoky atmosphere as twinkling lights nestled between traceries of magma.\\
Check:\\
Average [\df\df] \textbf{Knowledge (Outer Rim)} to gain information about Sullust and/or Average [\df\df] \textbf{Knowledge (Xenology)} for information about Sullustans. Extra information can be given for advantages, misinformation for disadvantages. Success provides information in the first few bullets.\\

On success:
\begin{itemize}
    \item Volcanic world on inner edge of Outer Rim with a noxious surface of ash, lava, and toxic gases.  
    \item Breath masks needed on surface when volcanoes active.
    \item Native Sullustans evolved from rodents in an underground ecosystem and have a natural sense of direction.
%    \item \advantage Descended from rodents, Sullustans have a natural sense of direction, which serves them well in the dark caverns and twisting tunnels they call home.
    \item \advantage  Sullustan engineering has tamed their hostile environment somewhat. Their cities have shunts that redirect lava to the surface after collecting geothermal energy from it.
   \item \advantage SoroSuub Corporation, one of the largest companies in the galaxy, is based on Sullust. It does galactic exports but also dominates the local economy as manufacturers, suppliers, and employers.  It has lucrative Imperial contracts.%has a monopoly on the manufacture and sale of consumer goods on Sullust. It employs ninety percent of Sullust’s workforce. It is a vertical monopoly; it mines, designs, manufactures, transports, and runs stores. It also has lucrative Imperial contracts.
	\item \advantage SoroSuub disbanded the Sullustan Council in 2 BBY via corporate takeover.
    \item \advantage Sullust is a source of fuel for the Imperial military and essential mining and manufacturing center for the Empire.  Sullustans, however, don't count as Imperial citizens due to their planet's vassal status.  It was on the side of the Confederacy of Independent Systems during the Clone Wars.
%    \item \advantage Was on the side of the Confederacy of Independent Systems during the Clone Wars. Reduced to vassal status after the Clone Wars and a source of fuel for the Imperial Military, Sullust became an essential mining and manufacturing center for the Empire. Sullustans don’t even count as citizens in the Empire.
    \item \advantage For years, Sullust remained relatively peaceful as workers accepted the stability offered under the Empire's reign. However, the Cobalt Laborers' Reformation Front steadily began to increase in numbers, sending letters to the Imperial governor demanding better working conditions and increased local autonomy. In response, the Empire detained roughly eighty percent of those in the organization and those it deemed most radical.
    \item \triumph Stan Tevv, former Galactic Senator, became SoroSuub CEO following the Battle of Yavin.  He has a secret agreement with the Rebel Alliance but officially claims neutrality.
    \item \threat Alter a minor detail in one of the above extra facts
    \item \despair Alter a major detail (like Sullustans planning to betray the Rebels)
\end{itemize}

    \subsection{Landing on Sullust}
Check:\\
The Comms tower contacts all incoming vessels.  Provided the party has legitimate credentials, this should present no problems.

\begin{quoting}
State the nature of your business in Pinyumb and transmit your crew identifications.
\end{quoting}
If necessary, PCs can make a Average [\df\df] check using the social skill of their choice to circumvent any unpleasantness at this point.  To expand this encounter, you can use the Spaceport Security or Administrator adversaries (Edge of the Empire, page 398).
\begin{quoting}
Responding with flight path and docking permit.\\
Cleared for three-day tourist/work visa.\\
Should you choose to remain longer, you’ll need to renew your visa.
\end{quoting}

Description:\\
The black swirling face of Sullust takes up more and more of the view as your craft angles towards the surface. The ship emerges from the ash cloud, headed directly for a yawning opening in the planet’s surface. There’s a magnetic shield across the entry, presumably to maintain the integrity of the atmosphere within: the ship passes through without incident. Inside, lights on each side of the tunnel guide you deeper into the planet’s crust. Other craft fly alongside you, both coming and going. The intake tunnel opens up into a hangar with your assigned berth.

\subsection{Hanger}
Check:\\
Average [\df\df] Perception to assess the security/threat level when the airlock opens.
\begin{itemize}
    \item \success It’s just crews loading and unloading, no one paying any special attention to the PCs. The few guards standing around near the exits are wearing corporate uniforms, not stormtrooper armor. If there’s an Imperial presence here, it is light.
    \item \failure PCs think they have attracted attention of local security
    \item \threat A detail that suggests heavier Imperial presence
\end{itemize}

\subsection{Turbolift down to Pinyumb}
Description:\\
Two guards (human and Sullustan) give you a bored once-over as you approach the lift. The lift itself is a wonder. Its walls are transparisteel, giving them a full view of the city as they’re lowered down to ground level. Pinyumb is built in a vast underground cavern, and its artificial towers rise like stalagmites, the tallest of them actually supporting the cave ceiling. The lights of the residences within glimmer in the perpetual twilight, their reflections sparkling like stars from the obsidian crystals overhead. Some kind of dusk-winged creatures wheel in the shadowy heights.

As the PCs step off the lift, it’s into a commercial district. Neon signs and flashing holos advertise shops and services to the foot traffic and speeders weaving among the towers. It's possible players may wish to return here later for supplies or information gathering.

The party has residential address for Aadlan Myder.

\section{Encounter: Not Home}

A Sullustan woman (Nuev) answers at apartment. There’s a milky film over her black eyes; she’s crying. Beyond here the PCs can she that there’s a group of Sullustans here, a whole family supporting each other emotionally. 
\begin{quoting}
You’ve come to say goodbye to Aadlan? 
\end{quoting}

Nuev is too upset to talk, but a younger Sullustan woman (Zien, a teenager) will animatedly clarify that he’s not leaving or sick.
\begin{quoting}
He’s being murdered.
\end{quoting}

\subsection{Talking with Zien}
Zien blames Cobalt Front for this emergency. SoroSuub is calling it an accident, but there’s a saying on Sullust, ``Accidents can happen to anyone, but they always happen to troublemakers.''\\
Check:\\
Social checks to interact with Zien are \df\df. If Coercion is tried, that gets \setback added, as she does not like being bossed around. If Negotition is tried in order to get payment for the rescue, that will earn a cinematic punch to the nose from her. 

On \success or \failure, Zien will provide information, but her demeanor will be affected by the result. With \success, she will also provide supplies. She does not want to get her mother's hopes up, though, so she keeps the conversation quiet. 

What is the current sitation?\\
\begin{quoting}
My father and his entire crew were trapped inside a side tunnel when a magma vent diverted behind them. They don’t have the equipment they’d need to get out, and all SoroSuub will say is that ‘rescue efforts are continuing.’ But my father says there’s been no contact at all from his superiors. Sometimes his comm signal still gets through. They’re going to die of thirst down there.
\end{quoting}

What kind of equipment is needed? \\
\begin{quoting}
Massive earth-movers, industrial grade lava dams, and the kind of specialized expertise required to reroute an active magma flow without melting the whole passage and drowning the miners in fire. Only SoroSuub could do it. But they won’t.
\end{quoting}

Can't they get out on their own?  \\
\begin{quoting}
They do have filtration masks, so toxins aren’t an issue. And there are vents to the surface. But they have no climbing gear and if they did, they’d just be stranded on the surface.
\end{quoting}

Rescue by ship? \\
\begin{quoting}
Volcanoes are constantly spewing ash and gases and sometimes magma. Visibility is near zero and lava geysers are completely unpredictable.  But you look like you could handle that.
\end{quoting}

Zien can provide climbing gear and target coordinates if PCs agree to mount the rescue. Filtration masks with cheek-lamps, rope, canteens of water, harnesses, etc. If the PCs failed their social check with her, they can head to the commercial district to acquire this equipment. One set of climbing gear is 50 credits. The special filtration masks with lights are 30 credits per PC.


\section{Finding Parking}

Reaching the coordinates from Zien requires weaving through gouts of flaming rock and threading in and out of black clouds of ash. But as the ship nears their target, the atmosphere grows denser. Before making the Pilot check to reach the coordinates, other PCs can aid the pilot.
Check: Plot path with sensors\\
Hard [\df\df\df] +\setback  \textbf{Perception}
\begin{itemize}
\item \success Each uncanceled \success removes a \setback for difficult terrain
\end{itemize}
Check:\\
A Force-sensitive could try Foresee to help with lava spouts. As they extend their feelings, starting nearby, there is the glow around each of their organic companions, something positive about each shining with a brilliant light. Potentially have that player identify what that is for each of the other characters, such as: compassion, courage, determination, or a drive to protect.

Stretching out from there: \\
Around them and under them Sullust is a much more complicated presence. The planet is in pain. Not so much from its ceaseless volcanic activity—there’s a kind of balance in that ever-changing instability, a rhythm to the chaos. But the factories that riddle its surface and the mines that bore into its depth are upsetting its own fragile harmony-in-motion, and the people—the people are exhausted, overworked, impoverished, and afraid. The Force here is not in balance.

A successful check removes the \setback\setback\setback for difficult terrain from the Piloting check.  Note: this same experience can be inserted elsewhere in the adventure the first time a PC tries to use the Force.\\
Check: Reach coordinates\\
\textbf{Piloting} \difficulty per Silhouette + \stb\stb\stb + Handling 

\begin{itemize}
	\item \advantage Gain familiarity with the timing of lava eruptions.  +\boost on next Pilot check or social interaction with a Sullustan.
	\item \failure Landing spot gets blown up by column of surging magma. Have to find alternate landing spot.
	\item \threat \threat A lava eruption strikes the ship.  Suffer 2 system strain.
	\item \despair Lava briefly engulfs the ship, causing a critical injury.
\end{itemize}
Check: Land\\
Once at the coordinates, another Piloting check is required to land without damaging the ship.
\difficulty per Silhouette + \stb\stb + Handling \textbf{Pilot}
\begin{itemize}
    \item \failure Hull trama
    \item \threat System strain
	\item \despair Critical
\end{itemize}
Check: Leads\\
Hard [\df\df\df] \textbf{Perception} or \textbf{Computers} +\setback to use the ships sensors to look for life signs (or use the Force). Doing this now will grant +\boost on the \textbf{Survival} check to come.

\section{Underground Travel}

The PCs exit the ship into hot, sulfurous air, with ash and smoke blackening it. They’ll take +\setback from heat to rolls if they’re out too long. If they lack filter masks, each PC takes one wound ``per round'' (in structured play, else as narratively appropriate).\\
Check: Climb below the surface\\
Hard [\df\df\df] \textbf{Athletics} +\boost for climbing gear
\begin{itemize}
  \item \threat\threat Accidentally unhook filter mask, suffering 1 wound that ignores soak
  \item \despair Slip and suffer a Short Fall (10 W/10 S) onto hard rock.  [\df\df] Coordination or Athletics to mitigate, as per Edge of the Empire, page 215.
\end{itemize}
Check: Environment wearing you down?\\
If two or more PCs failed their Athletics check for climbing, the party has taken too long to make it down the shaft. Every PC must make Easy [\df] \textbf{Resilience}. On a \failure that PC now suffers a Critical injury, Heat Exhaustion: +\setback on rolls until back in the cool confines of the ship. This condition can be removed with a Easy [\df] \textbf{Medicine} check.

After the climb, the party is now underground and can search for the missing Sullustans.\\
Check: Plot a course\\
Average [\df\df] \textbf{Survival} +\boost for sensor aid if available.\\
Check: Environment wearing you down?\\
If this \textbf{Survival} check fails, it costs the party more time. Every PC must make an Easy [\df] \textbf{Resilience} check or suffer Heat Exhaustion as per above.

As the PCs travel underground, the planet shifts and groans around them. The distant, arrhythmic hiss of venting steam never entirely ceases. Even through the masks, the air smells terrible. Eventually, they come upon a narrow crack that they must squeeze through.\\
Check:\\
Average [\df\df] \textbf{Coordination} to squeeze through a crack to get into a side passage. \boost for narratively small PCs (all Silhouette 0 characters, humans with a slighter frame, for example), \setback for larger PCs (all Silhouette 2+ characters, Wookies, brawny humans, etc).

Drutash scarabs (based on Green Bug Swarm from Beyond the Rim, page 44) attack when they’re in the crack!  Depending on the number of PCs, there may be one behind and one behind.\\
Check:\\
Hard [\df\df\df] \textbf{Xenology} to recognize Drutash Scarabs.  They are dangerous but also considered tasty delicacies.\\
Check:
Average [\df\df] \textbf{Survival} to harvest scarabs while avoiding poison sacs.  Success earns 100 credits + 10 per \advantage.

\npc{Drutash Scarab Swarm [Rival]}{1}{4}{1}{1}{1}{1}{20}{1}
\npccontinued{0}{0}{
\begin{GenesysTable}{ll}
    Skills & \\
       Brawl 2 &  Coordination 2
\end{GenesysTable}\\
\begin{GenesysTable}{ll}
    Abilities & \\
       Swarm & Unless attacked with a Blast or Burn quality\\
       & (regardless of if it is activated), halve damage \\
       & before applying soak\\
       Venomous Bite & If Swarm successfully hits a target, they \\
       &        must make an Average [\df\df] \textbf{Resilience} \\
       &        check or be disoriented (+\stb) for 3 rounds.\\
\end{GenesysTable}\\
}

After the crack, the passage is wider and tends uphill.

\section{The Miners}

The tunnel comes to a ledge, opening up into a small circular cavern. Main tunnel continues on the other side, angling back up to the surface. Miners are below under a collapsed section. The opening is no wider than an arm.\\
Check:\\
Average [\df\df] \textbf{Perception} or Hard [\df\df\df] \textbf{Vigilance} to see flash of miners’ lamp lights down through the crack.

\npc{Aadlan Myder [Rival]}{2}{3}{3}{1}{2}{3}{12}{3}
\npccontinued{0}{0}{
\begin{GenesysTable}{ll}
    Skills & \\
        Athletics 1 & Mechanics 3 \\
        Cool 1 & Perception 2 \\
        Leadership 1 & Streetwise 1 \\
\end{GenesysTable}\\
}

Miners will be suspicious.  Any interaction is first met with the question, ``Are you from SoroSuub?''\\
Check: Reassure miners\\
Hard [\challenge\df\df\stb\stb] \textbf{Charm} to convince Aadlan Myder of the PCs' good intentions. 
\begin{itemize}
	\item +\boost if PC making check is Sullustan.
	\item  +\boost if PC mentions they were sent by Zien Myder.
	\item  +\boost if PC mentions the Rebellion or Cobalt Front.
	\item  +\boost if PC sends down rehydration or similar supplies.
\end{itemize}
\begin{itemize}
	\item \advantage The miners have some useful equipment that combined with the PCs' gear, can help escape.
	\item \threat The miners are especially weakened and have to be carried out.
	\item \threat\threat The miners need immediate medical attention.
	\item \despair A minion group of miners is hallucinating from dehydration and thinks the PCs are attacking them.
\end{itemize}

\npc{Sullustan Miner [Minion]}{2}{3}{2}{1}{2}{2}{5}{2}
\npccontinued{0}{0}{
\begin{GenesysTable}{ll}
    Skills & \\
        Athletics & \\
        Mechanics &\\
        Perception & \\
\end{GenesysTable}\\
\begin{GenesysTable}{ll}
    Abilities & \\
       Tools & +2 dmg, Melee, Engaged, Crit 5\\
        & Improvised (+\threat), Inferior (+\setback) \\
\end{GenesysTable}\\
}

The miners are Aadlan Myder and four other Sullustans (Cars Mun, Truvv Nryll, Fri Anb, Omo Riirs). Their most immediate need is water. If the PCs explain that they are here to guide the miners back to the surface, there will be some protests and objections because of how unstable that is. Aadlan will try to calm them down, and an Average [\df\df] \textbf{Leadership} check from a PC would be helpful.

The PCs must widen the opening to get the miners out. Possible options are explosives (including overloaded weapons) or Force Move. The miners may have some explosives that can be reeled up to the PCs to place. Support any creative approaches that the party comes up with.

A Force-sensitive PC will have a sense of unease, that they need to hurry. When miners are starting to be pulled up, All PCs can make a Hard [\df\df\df] \textbf{Vigilance} check to sense rumbling under their feet.

The miners start to panic and say they need to hurry. There is a rockrender on the way, drawn by the explosion (or by the flip of a Dark Side point). When the last miner is on the way up, it breaks through into the cavern where the miners had been.\\
\begin{quoting}
Breaking into the cavern is a giant beast—reddish, quadrupedal, with enormous curved talons the size of a human body, and a set of tusks to match. It shakes off the rubble it created with its own entrance and lifts its head, as if sniffing. The last of the miners is dangling only about a meter above the beast. 
\end{quoting}
Check:\\
Hard [\df\df\df] \textbf{Knowledge (Xenology)} reveals that the Rockrender is not the danger.  The paths it opens are, as lava flows quickly through those.

\npc{Rockrender [Rival]}{6}{3}{1}{2}{1}{1}{60}{15}
\npccontinued{0}{0}{
\begin{GenesysTable}{ll}
    Skills & \\
       Brawl 2 & Coordination 2
\end{GenesysTable}
\begin{GenesysTable}{ll}
    Abilities & \\
        Burrow & Average [\df\df] \textbf{Coordination} to dig into the rock to escape \\
       Sharp claws & +2 brawl damage \\
\end{GenesysTable}
}
   The rockrender is really just passing through and will not attack anyone unless provoked.  See \url{https://starwars.fandom.com/wiki/Rockrender} for a reference.

Behind the creature, in the passage it made, there’s a dull orange glow that’s getting steadily brighter. The magma is rising.

The miners are all exhausted/dehydrated and will move slowly. The PCs could try using explosives to collapse the tunnel behind them to slow the magma. Going back the way they came means heading downhill, and the magma will follow them that way. They can try going uphill on the other side of the cavern.

The PCs need 3 successful Hard [\df\df\df] \textbf{Athletics} checks to get all the miners out before the chamber floods with magma.  Failure to do so means either a PC suffers a critical injury to rescue a miner, or the Sullustans lose trust in the PCs (and possibly the Rebel Alliance).

That is the end of the action. Of course, you can have PCs make a Piloting check to get back to Pinyumb instead of scene swiping if you wish.

\section{Aftermath}

Aadlan will want medical personnel to meet them in Pinyumb, rather than have any private treatment from the PCs. He wants them to call ahead to the comm operator there.\\
\begin{quoting}
Make it public. They can’t assassinate me without plausible deniability.  You’ve seen what they’ll do to silence the Cobalt Laborer’s Reformation Front.
\end{quoting}

Zien, Nuev, and the rest of the Myder family will be waiting at Pinyumb to greet the rescued miners. Some media/reporters will be on the scene as well. PCs can slip the box of transponders in among all the gear they are returning. SoroSuub representatives led by Ub Nub will also be present, acting relieved at how safe the workers are and trying to put a positive spin on things.

The PCs may be content to just leave at that point. However, if they are looking for someone to blame or to blackmail for SoroSuub's lack of effort, Ub Nub is a reasonble target.

\npc{Ub Nub}{2}{2}{2}{3}{3}{3}{12}{2}
\npccontinued{0}{0}{
\begin{GenesysTable}{ll}
    Skills & \\
       Charm 2 & Deception 3\\
       Coercion 1 & Negotiation 4\\
       Cool 2 & \\
\end{GenesysTable}
\begin{GenesysTable}{ll}
    Abilities & \\
       Plausible Deniability 1 & remove \setback from all \\
       & Coercion and Deception checks \\
       Nobody's Fool 2 & upgrade diffuclty of Charm, Coercion,\\
       & and Deception checks made against\\
       &this target twice\\
\end{GenesysTable}
}

Recommended XP: 5 per major encounter, so probably 15-20 depending on how much time the party spent on various aspects of the adventure.

\end{multicols}
\end{document}

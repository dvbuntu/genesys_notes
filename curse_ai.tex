\documentclass{book}

\usepackage[dvipsnames, table]{xcolor}
%\usepackage{swrpg}
\usepackage{genesys}
\usepackage{xspace} % fix space issues after text macro
\usepackage{hyperref}
%\geometry{paperheight=13in}
\usepackage{multicol}
\usepackage[noorphans,font=itshape,indentfirst=false,rightmargin=0pt]{quoting}

%\swrpgbackground
\newcommand{\spa}{\hskip 0.3em }

% dice
\newcommand{\df}{\DifficultyDie }
\newcommand{\stb}{\SetbackDie }
\newcommand{\ch}{\ChallengeDie }
\newcommand{\abd}{\AbilityDie }
\newcommand{\bbb}{\BoostDie }
\newcommand{\pd}{\ProficiencyDie }

\renewcommand\Characteristics[6]{
\noindent\hfill
\charPart{BRAWN}{#1}\kern-.1em
\charPart{AGILITY}{#2}\kern-.1em
\charPart{INTELLECT}{#3}\kern-.1em
\charPart{CUNNING}{#4}\kern-.1em
\charPart{WILLPOWER}{#5}\kern-.1em
\charPart{PRESENCE}{#6}%
\hfill\null}

\newcommand{\npc}[9]{
    \begin{GenesysTable}{l} {\Large #1 } \end{GenesysTable} \\ %name
    \Characteristics{#2}{#3}{#4}{#5}{#6}{#7}\\
    \vskip -1em
    \Derived{Wounds}{#8}
    \Derived{Soak}{#9}
}
\newcommand{\npccontinued}[3]{
    \DerivedSplit{Defense}{#1}{#2}{Melee}{Ranged}\\ % defense
    \vskip -2.5em
    #3 %description, abilities, etc
}

\newcommand{\nemesis}[9]{
    \begin{GenesysTable}{l} {\Large #1 [Nemesis] } \end{GenesysTable} \\ %name
    \Characteristics{#2}{#3}{#4}{#5}{#6}{#7}\\
    \vskip -1em
    \Derived{Wounds}{#8}\hskip-.1em
    \Derived{Strain}{#9}\hskip-.1em
}
\newcommand{\nemesiscontinued}[4]{
    \Derived{Soak}{#1}\hskip-.1em
    \DerivedSplit{Defense}{#2}{#3}{Melee}{Ranged}\\ % defense
    \vskip -2.5em
    #4 %description, abilities, etc
}

\newcommand{\ai}{{\color{cyan}Artificial Indigo}\xspace}



\title{
    The Curse of \ai \\
A DiceyStories.com Genesys Adventure\\
}
\author{Author: Danielle Van Boxel}
\date{\today}


\begin{document}
\maketitle


\begin{multicols*}{2}
\section{Gemini Mountains}

\emph{Present Day, 1991, the small town of Gemini Mountains.  Can a struggling company of paranormal investigators locate a lost lover while evading corporately-cursed clothing?  Tune in to find out!  Thursdays 8PM Eastern/7PM Central on ABC.  PG-13.}\\

    The Curse of \ai is a Genesys adventure for 3-5 Player Characters (PCs) with a typical playtime of 3 hours.  Inspired by 1990s supernatural television like \emph{Twin Peaks}, \emph{Buffy The Vampire Slayer}, \emph{The X-Files}, and others, the setting is mundane except the PCs and select adversaries are familiar with supernatural forces that lurk beneath public awareness.

\section{Adventure Overview}

    The PCs are a band of paranomal investigators/exterminators hired by Andre to search for his missing boyfriend, Breaux.  In doing so, they uncover Breaux's connection to Advanced Biotech Corporation (ABC) and the heavily marketed \ai clothing product line.  The PCs will find that deploying \ai has some serious consequences for everyone involved.\\
\\
\emph{Goal}: Discover what happened to Breaux and uncover the Curse of \ai.\\
\\
\emph{Challenge}: (spoiler) \ai is a supernatural clothing product that requires skill ``donations'' to grant those skills to the wearer.  Breaux fell into debt using their products unsuccessfully, so ABC has ``repossessed'' him to work that off.  \ai clothes also tend to go haywire and operate without a wearer.\\
\\
\emph{Narrative Reminder}: (spoiler) \ai adversaries show a degree of autonomy and malice, but not \emph{sentience}.  They display no pain when damaged nor joy when successful.  Play them similarly to modern artificial intelligence models, which can act chaotically and without regard for logic.

\subsection{Encounter List}
\begin{enumerate}
    \setlength\itemsep{-.5em}
    \item Title Sequence (narrative/character intro)
    \item Investigators For Hire (social)
    \item Asking Around Town (information/investigation)
    \item Haunted Workshop (combat/investigation)
    \item Entering ABC Facility (social/stealth)
    \item Hallway Interdiction (combat/stealth)
    \item Breaux's Sacrifice (combat/social)
    \item Aftermath
\end{enumerate}

\section{Title Sequence}

To help players visualize the setting and get into character, start with an almost literal title sequence of a news piece about/ABC ad for \ai.  Read the italicized text:

\emph{The camera zooms in on a reporter interviewing the winner of the local Gemini Mountains marathon, an unfamiliar woman wearing bright blue leggings.  She is frenetic, unable to stand still as she explains, ``A week ago I could barely run a 5k, but now I'm a champion runner.  Thanks \ai!''  An awkward smile crosses her face as she stutters a goodbye and starts a victory lap, seemingly unable to stop her legs from running.}

\emph{The camera briefly pans over to another marathoner, exhausted and drained, leaning against a wall.  You recognize her as the previous three-time marathon winner in town.  Smash cut to the character intro sequences.}

Ask one of the PCs to briefly describe their introductory sequence as though it were television show.  This should be a brief shot of the character doing something key to their concept, without dialogue, and unrelated to any specific adventure.  After they describe their character, ask the next player and so on.

\emph{Example Introduction 1}: Candra the witch is sitting cross-legged in a pentagram.  Five candles sit unlit at each point of the star.  Reflected in Candra's eyes, we see a candle ignite.

\emph{Example Introduction 2}: Sebastian the engineer is typing furiously at a bulky 1990s laptop computer.  The camera rotates around the scene to show ``Ghost Program Activated'' on the screen as Sebastian grins.

After all characters have introduced themselves, the show title appears, displaying,``The Curse of \ai''.  You may optionally ``cut to a commercial break'' here to further immerse players in the television aesthetic.

\section{Investigators For Hire}

Andre, a well-dressed lawyer, enters the PCs' paranormal extermination office during his lunch break.  If the PCs have not yet named their business, ask them for a name now.  He is clearly distressed or in a hurry as he steps inside.

    Andre wants to hire the PCs to find and bring back his missing ``roommate'' and ``friend'', Breaux (pronounced "bro").  He can pay a significant amount up front (the PCs desperately need the income, or if part of a long-running campaign, some other information/object/favor from Andre) and is looking to close a deal quickly.  Andre explains that he needs paranormal investigators because Breaux was last seen heading to his haunted workshop several days ago.  Andre is terrified of ghosts and refuses to go there.

    PCs should first attempt to elicit more details from Andre before making a \textbf{Negotiation} check to close the deal.  Many different approaches are reasonable and offer different benefits (narrative or mechanical), so here are few examples to explain Andre's situation.

\vspace{1em}
\begin{GenesysTable}{p{1.5in}p{2in}}
    Approach & \\
    \df\df\stb \textbf{Knowledge (Mundane)} to understand Andre/Breaux's relationship & Andre's concern goes beyond roommates.  They are boyfriends.+\bbb to \textbf{Negotiation}\\
    \df\df \textbf{Leadership} to ask for more detail & Breaux has been working on an art project at the workshop and making many purchases at a local hardware store.\\
    \df\df \textbf{Perception} to observe Andre & He keeps glancing at his watch and avoids looking at paranormal objects in the shop.  +\bbb to \textbf{Negotiate}\\
\end{GenesysTable}

Let each PC make one such informational roll (if they want).  Then the group must make a single \ch\df\df\textbf{Negotiation} check to sign the investigation contract with Andre.  They're going to get the job either way, but the results can influence immediate (and if appropriate, campaign level) benefits/hindrances:

\begin{itemize}
    \setlength\itemsep{-.5em}
        \item \Success: The PCs get out of immediate financial distress by taking this job.
        \item \Failure: Andre signs the contract, but it's not enough to keep the business afloat long.  +\stb to all social interactions with townspeople for various debts stil owed.
        \item \Advantage\Advantage: Every two advantages constitute profit they can use to use for +\bbb on future social checks or longer term needs.
        \item \Threat\Threat: Andre is concerned about secrecy and insists that the PCs avoid disclosing that they are searching for Breaux.
        \item \Triumph: Andre gives the PCs one of Breaux work gloves. It is bright blue and shows the logo of ABC's \ai.
        \item \Despair: Andre also hires a more traditional private investigator that will compete and interfere with the PCs' efforts.  If necessary, use stats similar to the PCs, except substitute more mundane skills for anything supernatural.
    \end{itemize}

After signing the contract, Andre leaves quickly to get back to work.  The PCs have picture of Breaux and Andre together from a mall photobooth kiosk as well as the address for Breaux's haunted workshop.  While Andre cannot immediately go with the PCs, if they find strong evidence that Breaux is at the ABC facility, he may accompany them there if pressed with a social check.  He does not have his own stat block but rather can upgrade PCs social and \textbf{Knowledge (Mundane)} checks with his legal knowledge.

\section{Asking Around Town (optional)}

PCs are likely to seek more information about Breaux's whereabouts from various townspeople or other investigations.  These are a few possibilities and details you can share if they are successful.  If time is limited, you may wish to skip this encounter, rolling these details into information dispursed by Andre.

\section{Haunted Workshop}

\section{Entering ABC Facility}

\section{Hallway Interdiction (optional)}

\section{Breaux's Sacrifice}

\section{Aftermath}

    This is a narrative epilogue for the ``viewers'' of the television program, with the PCs remaining unaware of these details (until a possible future adventure).  If a Trenchcoat escaped from the final combat, the camera follows it fleeing to an administrative ABC building.  Otherwise, a fresh Trenchcoat fills a similar role.  In a dimly-lit executive office, the Trenchcoat pantomimes the PCs' rescue of Breaux to the back of an expensive high-backed office chair.

    The chair slowly rotates, revealing a stern woman in business-casual attire (notably, \emph{not} bright blue like \ai products).  Showing little emotion, she calmly snaps her fingers.  The trenchcoat bursts into flame, continuing to flail even as it turns to ash.  The camera zooms out and the viewers see a wall of supernatural artifacts.  As it pulls beyond the office door, the executive's name placard becomes visible, identifying her as, ``Cillarma Tepes, Chief Arcane Officer.''

If you continue in this campaign setting, Cillarma can be a long-term adversary, wielding both mundane corporate and supernatural abilities against the PCs and their allies.  ``Cillarma'' anagrams to ``Carmilla'', one of the earliest vampires popularized in English fiction.  Adapt her background (and name) as needed to your campaign.

\end{multicols*}
\end{document}
